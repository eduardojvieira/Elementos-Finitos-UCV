
% Default to the notebook output style

    


% Inherit from the specified cell style.




    
\documentclass{article}

    
    
    \usepackage{graphicx} % Used to insert images
    \usepackage{adjustbox} % Used to constrain images to a maximum size 
    \usepackage{color} % Allow colors to be defined
    \usepackage{enumerate} % Needed for markdown enumerations to work
    \usepackage{geometry} % Used to adjust the document margins
    \usepackage{amsmath} % Equations
    \usepackage{amssymb} % Equations
    \usepackage{eurosym} % defines \euro
    \usepackage[mathletters]{ucs} % Extended unicode (utf-8) support
    \usepackage[utf8x]{inputenc} % Allow utf-8 characters in the tex document
    \usepackage{fancyvrb} % verbatim replacement that allows latex
    \usepackage{grffile} % extends the file name processing of package graphics 
                         % to support a larger range 
    % The hyperref package gives us a pdf with properly built
    % internal navigation ('pdf bookmarks' for the table of contents,
    % internal cross-reference links, web links for URLs, etc.)
    \usepackage{hyperref}
    \usepackage{longtable} % longtable support required by pandoc >1.10
    \usepackage{booktabs}  % table support for pandoc > 1.12.2
    \usepackage{ulem} % ulem is needed to support strikethroughs (\sout)
    

    
    
    \definecolor{orange}{cmyk}{0,0.4,0.8,0.2}
    \definecolor{darkorange}{rgb}{.71,0.21,0.01}
    \definecolor{darkgreen}{rgb}{.12,.54,.11}
    \definecolor{myteal}{rgb}{.26, .44, .56}
    \definecolor{gray}{gray}{0.45}
    \definecolor{lightgray}{gray}{.95}
    \definecolor{mediumgray}{gray}{.8}
    \definecolor{inputbackground}{rgb}{.95, .95, .85}
    \definecolor{outputbackground}{rgb}{.95, .95, .95}
    \definecolor{traceback}{rgb}{1, .95, .95}
    % ansi colors
    \definecolor{red}{rgb}{.6,0,0}
    \definecolor{green}{rgb}{0,.65,0}
    \definecolor{brown}{rgb}{0.6,0.6,0}
    \definecolor{blue}{rgb}{0,.145,.698}
    \definecolor{purple}{rgb}{.698,.145,.698}
    \definecolor{cyan}{rgb}{0,.698,.698}
    \definecolor{lightgray}{gray}{0.5}
    
    % bright ansi colors
    \definecolor{darkgray}{gray}{0.25}
    \definecolor{lightred}{rgb}{1.0,0.39,0.28}
    \definecolor{lightgreen}{rgb}{0.48,0.99,0.0}
    \definecolor{lightblue}{rgb}{0.53,0.81,0.92}
    \definecolor{lightpurple}{rgb}{0.87,0.63,0.87}
    \definecolor{lightcyan}{rgb}{0.5,1.0,0.83}
    
    % commands and environments needed by pandoc snippets
    % extracted from the output of `pandoc -s`
    \providecommand{\tightlist}{%
      \setlength{\itemsep}{0pt}\setlength{\parskip}{0pt}}
    \DefineVerbatimEnvironment{Highlighting}{Verbatim}{commandchars=\\\{\}}
    % Add ',fontsize=\small' for more characters per line
    \newenvironment{Shaded}{}{}
    \newcommand{\KeywordTok}[1]{\textcolor[rgb]{0.00,0.44,0.13}{\textbf{{#1}}}}
    \newcommand{\DataTypeTok}[1]{\textcolor[rgb]{0.56,0.13,0.00}{{#1}}}
    \newcommand{\DecValTok}[1]{\textcolor[rgb]{0.25,0.63,0.44}{{#1}}}
    \newcommand{\BaseNTok}[1]{\textcolor[rgb]{0.25,0.63,0.44}{{#1}}}
    \newcommand{\FloatTok}[1]{\textcolor[rgb]{0.25,0.63,0.44}{{#1}}}
    \newcommand{\CharTok}[1]{\textcolor[rgb]{0.25,0.44,0.63}{{#1}}}
    \newcommand{\StringTok}[1]{\textcolor[rgb]{0.25,0.44,0.63}{{#1}}}
    \newcommand{\CommentTok}[1]{\textcolor[rgb]{0.38,0.63,0.69}{\textit{{#1}}}}
    \newcommand{\OtherTok}[1]{\textcolor[rgb]{0.00,0.44,0.13}{{#1}}}
    \newcommand{\AlertTok}[1]{\textcolor[rgb]{1.00,0.00,0.00}{\textbf{{#1}}}}
    \newcommand{\FunctionTok}[1]{\textcolor[rgb]{0.02,0.16,0.49}{{#1}}}
    \newcommand{\RegionMarkerTok}[1]{{#1}}
    \newcommand{\ErrorTok}[1]{\textcolor[rgb]{1.00,0.00,0.00}{\textbf{{#1}}}}
    \newcommand{\NormalTok}[1]{{#1}}
    
    % Additional commands for more recent versions of Pandoc
    \newcommand{\ConstantTok}[1]{\textcolor[rgb]{0.53,0.00,0.00}{{#1}}}
    \newcommand{\SpecialCharTok}[1]{\textcolor[rgb]{0.25,0.44,0.63}{{#1}}}
    \newcommand{\VerbatimStringTok}[1]{\textcolor[rgb]{0.25,0.44,0.63}{{#1}}}
    \newcommand{\SpecialStringTok}[1]{\textcolor[rgb]{0.73,0.40,0.53}{{#1}}}
    \newcommand{\ImportTok}[1]{{#1}}
    \newcommand{\DocumentationTok}[1]{\textcolor[rgb]{0.73,0.13,0.13}{\textit{{#1}}}}
    \newcommand{\AnnotationTok}[1]{\textcolor[rgb]{0.38,0.63,0.69}{\textbf{\textit{{#1}}}}}
    \newcommand{\CommentVarTok}[1]{\textcolor[rgb]{0.38,0.63,0.69}{\textbf{\textit{{#1}}}}}
    \newcommand{\VariableTok}[1]{\textcolor[rgb]{0.10,0.09,0.49}{{#1}}}
    \newcommand{\ControlFlowTok}[1]{\textcolor[rgb]{0.00,0.44,0.13}{\textbf{{#1}}}}
    \newcommand{\OperatorTok}[1]{\textcolor[rgb]{0.40,0.40,0.40}{{#1}}}
    \newcommand{\BuiltInTok}[1]{{#1}}
    \newcommand{\ExtensionTok}[1]{{#1}}
    \newcommand{\PreprocessorTok}[1]{\textcolor[rgb]{0.74,0.48,0.00}{{#1}}}
    \newcommand{\AttributeTok}[1]{\textcolor[rgb]{0.49,0.56,0.16}{{#1}}}
    \newcommand{\InformationTok}[1]{\textcolor[rgb]{0.38,0.63,0.69}{\textbf{\textit{{#1}}}}}
    \newcommand{\WarningTok}[1]{\textcolor[rgb]{0.38,0.63,0.69}{\textbf{\textit{{#1}}}}}
    
    
    % Define a nice break command that doesn't care if a line doesn't already
    % exist.
    \def\br{\hspace*{\fill} \\* }
    % Math Jax compatability definitions
    \def\gt{>}
    \def\lt{<}
    % Document parameters
    \title{Elementos Finitos Ejercicio 4.6}
    
    
    

    % Pygments definitions
    
\makeatletter
\def\PY@reset{\let\PY@it=\relax \let\PY@bf=\relax%
    \let\PY@ul=\relax \let\PY@tc=\relax%
    \let\PY@bc=\relax \let\PY@ff=\relax}
\def\PY@tok#1{\csname PY@tok@#1\endcsname}
\def\PY@toks#1+{\ifx\relax#1\empty\else%
    \PY@tok{#1}\expandafter\PY@toks\fi}
\def\PY@do#1{\PY@bc{\PY@tc{\PY@ul{%
    \PY@it{\PY@bf{\PY@ff{#1}}}}}}}
\def\PY#1#2{\PY@reset\PY@toks#1+\relax+\PY@do{#2}}

\expandafter\def\csname PY@tok@gd\endcsname{\def\PY@tc##1{\textcolor[rgb]{0.63,0.00,0.00}{##1}}}
\expandafter\def\csname PY@tok@gu\endcsname{\let\PY@bf=\textbf\def\PY@tc##1{\textcolor[rgb]{0.50,0.00,0.50}{##1}}}
\expandafter\def\csname PY@tok@gt\endcsname{\def\PY@tc##1{\textcolor[rgb]{0.00,0.27,0.87}{##1}}}
\expandafter\def\csname PY@tok@gs\endcsname{\let\PY@bf=\textbf}
\expandafter\def\csname PY@tok@gr\endcsname{\def\PY@tc##1{\textcolor[rgb]{1.00,0.00,0.00}{##1}}}
\expandafter\def\csname PY@tok@cm\endcsname{\let\PY@it=\textit\def\PY@tc##1{\textcolor[rgb]{0.25,0.50,0.50}{##1}}}
\expandafter\def\csname PY@tok@vg\endcsname{\def\PY@tc##1{\textcolor[rgb]{0.10,0.09,0.49}{##1}}}
\expandafter\def\csname PY@tok@m\endcsname{\def\PY@tc##1{\textcolor[rgb]{0.40,0.40,0.40}{##1}}}
\expandafter\def\csname PY@tok@mh\endcsname{\def\PY@tc##1{\textcolor[rgb]{0.40,0.40,0.40}{##1}}}
\expandafter\def\csname PY@tok@go\endcsname{\def\PY@tc##1{\textcolor[rgb]{0.53,0.53,0.53}{##1}}}
\expandafter\def\csname PY@tok@ge\endcsname{\let\PY@it=\textit}
\expandafter\def\csname PY@tok@vc\endcsname{\def\PY@tc##1{\textcolor[rgb]{0.10,0.09,0.49}{##1}}}
\expandafter\def\csname PY@tok@il\endcsname{\def\PY@tc##1{\textcolor[rgb]{0.40,0.40,0.40}{##1}}}
\expandafter\def\csname PY@tok@cs\endcsname{\let\PY@it=\textit\def\PY@tc##1{\textcolor[rgb]{0.25,0.50,0.50}{##1}}}
\expandafter\def\csname PY@tok@cp\endcsname{\def\PY@tc##1{\textcolor[rgb]{0.74,0.48,0.00}{##1}}}
\expandafter\def\csname PY@tok@gi\endcsname{\def\PY@tc##1{\textcolor[rgb]{0.00,0.63,0.00}{##1}}}
\expandafter\def\csname PY@tok@gh\endcsname{\let\PY@bf=\textbf\def\PY@tc##1{\textcolor[rgb]{0.00,0.00,0.50}{##1}}}
\expandafter\def\csname PY@tok@ni\endcsname{\let\PY@bf=\textbf\def\PY@tc##1{\textcolor[rgb]{0.60,0.60,0.60}{##1}}}
\expandafter\def\csname PY@tok@nl\endcsname{\def\PY@tc##1{\textcolor[rgb]{0.63,0.63,0.00}{##1}}}
\expandafter\def\csname PY@tok@nn\endcsname{\let\PY@bf=\textbf\def\PY@tc##1{\textcolor[rgb]{0.00,0.00,1.00}{##1}}}
\expandafter\def\csname PY@tok@no\endcsname{\def\PY@tc##1{\textcolor[rgb]{0.53,0.00,0.00}{##1}}}
\expandafter\def\csname PY@tok@na\endcsname{\def\PY@tc##1{\textcolor[rgb]{0.49,0.56,0.16}{##1}}}
\expandafter\def\csname PY@tok@nb\endcsname{\def\PY@tc##1{\textcolor[rgb]{0.00,0.50,0.00}{##1}}}
\expandafter\def\csname PY@tok@nc\endcsname{\let\PY@bf=\textbf\def\PY@tc##1{\textcolor[rgb]{0.00,0.00,1.00}{##1}}}
\expandafter\def\csname PY@tok@nd\endcsname{\def\PY@tc##1{\textcolor[rgb]{0.67,0.13,1.00}{##1}}}
\expandafter\def\csname PY@tok@ne\endcsname{\let\PY@bf=\textbf\def\PY@tc##1{\textcolor[rgb]{0.82,0.25,0.23}{##1}}}
\expandafter\def\csname PY@tok@nf\endcsname{\def\PY@tc##1{\textcolor[rgb]{0.00,0.00,1.00}{##1}}}
\expandafter\def\csname PY@tok@si\endcsname{\let\PY@bf=\textbf\def\PY@tc##1{\textcolor[rgb]{0.73,0.40,0.53}{##1}}}
\expandafter\def\csname PY@tok@s2\endcsname{\def\PY@tc##1{\textcolor[rgb]{0.73,0.13,0.13}{##1}}}
\expandafter\def\csname PY@tok@vi\endcsname{\def\PY@tc##1{\textcolor[rgb]{0.10,0.09,0.49}{##1}}}
\expandafter\def\csname PY@tok@nt\endcsname{\let\PY@bf=\textbf\def\PY@tc##1{\textcolor[rgb]{0.00,0.50,0.00}{##1}}}
\expandafter\def\csname PY@tok@nv\endcsname{\def\PY@tc##1{\textcolor[rgb]{0.10,0.09,0.49}{##1}}}
\expandafter\def\csname PY@tok@s1\endcsname{\def\PY@tc##1{\textcolor[rgb]{0.73,0.13,0.13}{##1}}}
\expandafter\def\csname PY@tok@kd\endcsname{\let\PY@bf=\textbf\def\PY@tc##1{\textcolor[rgb]{0.00,0.50,0.00}{##1}}}
\expandafter\def\csname PY@tok@sh\endcsname{\def\PY@tc##1{\textcolor[rgb]{0.73,0.13,0.13}{##1}}}
\expandafter\def\csname PY@tok@sc\endcsname{\def\PY@tc##1{\textcolor[rgb]{0.73,0.13,0.13}{##1}}}
\expandafter\def\csname PY@tok@sx\endcsname{\def\PY@tc##1{\textcolor[rgb]{0.00,0.50,0.00}{##1}}}
\expandafter\def\csname PY@tok@bp\endcsname{\def\PY@tc##1{\textcolor[rgb]{0.00,0.50,0.00}{##1}}}
\expandafter\def\csname PY@tok@c1\endcsname{\let\PY@it=\textit\def\PY@tc##1{\textcolor[rgb]{0.25,0.50,0.50}{##1}}}
\expandafter\def\csname PY@tok@kc\endcsname{\let\PY@bf=\textbf\def\PY@tc##1{\textcolor[rgb]{0.00,0.50,0.00}{##1}}}
\expandafter\def\csname PY@tok@c\endcsname{\let\PY@it=\textit\def\PY@tc##1{\textcolor[rgb]{0.25,0.50,0.50}{##1}}}
\expandafter\def\csname PY@tok@mf\endcsname{\def\PY@tc##1{\textcolor[rgb]{0.40,0.40,0.40}{##1}}}
\expandafter\def\csname PY@tok@err\endcsname{\def\PY@bc##1{\setlength{\fboxsep}{0pt}\fcolorbox[rgb]{1.00,0.00,0.00}{1,1,1}{\strut ##1}}}
\expandafter\def\csname PY@tok@mb\endcsname{\def\PY@tc##1{\textcolor[rgb]{0.40,0.40,0.40}{##1}}}
\expandafter\def\csname PY@tok@ss\endcsname{\def\PY@tc##1{\textcolor[rgb]{0.10,0.09,0.49}{##1}}}
\expandafter\def\csname PY@tok@sr\endcsname{\def\PY@tc##1{\textcolor[rgb]{0.73,0.40,0.53}{##1}}}
\expandafter\def\csname PY@tok@mo\endcsname{\def\PY@tc##1{\textcolor[rgb]{0.40,0.40,0.40}{##1}}}
\expandafter\def\csname PY@tok@kn\endcsname{\let\PY@bf=\textbf\def\PY@tc##1{\textcolor[rgb]{0.00,0.50,0.00}{##1}}}
\expandafter\def\csname PY@tok@mi\endcsname{\def\PY@tc##1{\textcolor[rgb]{0.40,0.40,0.40}{##1}}}
\expandafter\def\csname PY@tok@gp\endcsname{\let\PY@bf=\textbf\def\PY@tc##1{\textcolor[rgb]{0.00,0.00,0.50}{##1}}}
\expandafter\def\csname PY@tok@o\endcsname{\def\PY@tc##1{\textcolor[rgb]{0.40,0.40,0.40}{##1}}}
\expandafter\def\csname PY@tok@kr\endcsname{\let\PY@bf=\textbf\def\PY@tc##1{\textcolor[rgb]{0.00,0.50,0.00}{##1}}}
\expandafter\def\csname PY@tok@s\endcsname{\def\PY@tc##1{\textcolor[rgb]{0.73,0.13,0.13}{##1}}}
\expandafter\def\csname PY@tok@kp\endcsname{\def\PY@tc##1{\textcolor[rgb]{0.00,0.50,0.00}{##1}}}
\expandafter\def\csname PY@tok@w\endcsname{\def\PY@tc##1{\textcolor[rgb]{0.73,0.73,0.73}{##1}}}
\expandafter\def\csname PY@tok@kt\endcsname{\def\PY@tc##1{\textcolor[rgb]{0.69,0.00,0.25}{##1}}}
\expandafter\def\csname PY@tok@ow\endcsname{\let\PY@bf=\textbf\def\PY@tc##1{\textcolor[rgb]{0.67,0.13,1.00}{##1}}}
\expandafter\def\csname PY@tok@sb\endcsname{\def\PY@tc##1{\textcolor[rgb]{0.73,0.13,0.13}{##1}}}
\expandafter\def\csname PY@tok@k\endcsname{\let\PY@bf=\textbf\def\PY@tc##1{\textcolor[rgb]{0.00,0.50,0.00}{##1}}}
\expandafter\def\csname PY@tok@se\endcsname{\let\PY@bf=\textbf\def\PY@tc##1{\textcolor[rgb]{0.73,0.40,0.13}{##1}}}
\expandafter\def\csname PY@tok@sd\endcsname{\let\PY@it=\textit\def\PY@tc##1{\textcolor[rgb]{0.73,0.13,0.13}{##1}}}

\def\PYZbs{\char`\\}
\def\PYZus{\char`\_}
\def\PYZob{\char`\{}
\def\PYZcb{\char`\}}
\def\PYZca{\char`\^}
\def\PYZam{\char`\&}
\def\PYZlt{\char`\<}
\def\PYZgt{\char`\>}
\def\PYZsh{\char`\#}
\def\PYZpc{\char`\%}
\def\PYZdl{\char`\$}
\def\PYZhy{\char`\-}
\def\PYZsq{\char`\'}
\def\PYZdq{\char`\"}
\def\PYZti{\char`\~}
% for compatibility with earlier versions
\def\PYZat{@}
\def\PYZlb{[}
\def\PYZrb{]}
\makeatother


    % Exact colors from NB
    \definecolor{incolor}{rgb}{0.0, 0.0, 0.5}
    \definecolor{outcolor}{rgb}{0.545, 0.0, 0.0}



    
    % Prevent overflowing lines due to hard-to-break entities
    \sloppy 
    % Setup hyperref package
    \hypersetup{
      breaklinks=true,  % so long urls are correctly broken across lines
      colorlinks=true,
      urlcolor=blue,
      linkcolor=darkorange,
      citecolor=darkgreen,
      }
    % Slightly bigger margins than the latex defaults
    
    \geometry{verbose,tmargin=1in,bmargin=1in,lmargin=1in,rmargin=1in}
    
    

    \begin{document}
    
    
    \maketitle
    
    

    
    \begin{Verbatim}[commandchars=\\\{\}]
{\color{incolor}In [{\color{incolor}1}]:} \PY{c}{\PYZsh{} Se importan las librerías a utilizar}
        \PY{k+kn}{import} \PY{n+nn}{pandas} \PY{k+kn}{as} \PY{n+nn}{pd}
        \PY{k+kn}{import} \PY{n+nn}{numpy} \PY{k+kn}{as} \PY{n+nn}{np}
        \PY{k+kn}{from} \PY{n+nn}{\PYZus{}\PYZus{}future\PYZus{}\PYZus{}} \PY{k+kn}{import} \PY{n}{division}\PY{p}{,} \PY{n}{print\PYZus{}function}
        
        \PY{c}{\PYZsh{} Datos}
        \PY{n}{E} \PY{o}{=} \PY{l+m+mf}{200e9}              \PY{c}{\PYZsh{} Módulo elśtico del material [Pa]}
        \PY{n}{A} \PY{o}{=} \PY{l+m+mi}{250} \PY{o}{/} \PY{p}{(}\PY{l+m+mi}{1000} \PY{o}{*}\PY{o}{*} \PY{l+m+mi}{2}\PY{p}{)}  \PY{c}{\PYZsh{} Área transversal en [m2]}
        \PY{n}{F} \PY{o}{=} \PY{l+m+mf}{18e3}               \PY{c}{\PYZsh{} Fuerza aplicda en 1 [N]}
        
        \PY{c}{\PYZsh{} Coordenadas respecto a 1 [m]}
        \PY{c}{\PYZsh{} 1}
        \PY{n}{x\PYZus{}1} \PY{o}{=} \PY{l+m+mi}{0}
        \PY{n}{y\PYZus{}1} \PY{o}{=} \PY{l+m+mi}{0}
        
        \PY{c}{\PYZsh{} 2}
        \PY{n}{x\PYZus{}2} \PY{o}{=} \PY{o}{\PYZhy{}}\PY{o}{.}\PY{l+m+mi}{45}
        \PY{n}{y\PYZus{}2} \PY{o}{=} \PY{o}{.}\PY{l+m+mi}{6}
        
        \PY{c}{\PYZsh{} 3}
        \PY{n}{x\PYZus{}3} \PY{o}{=} \PY{o}{.}\PY{l+m+mi}{45} \PY{o}{+} \PY{o}{.}\PY{l+m+mi}{35}
        \PY{n}{y\PYZus{}3} \PY{o}{=} \PY{o}{.}\PY{l+m+mi}{6}
        
        \PY{c}{\PYZsh{} 4}
        \PY{n}{x\PYZus{}4} \PY{o}{=} \PY{o}{.}\PY{l+m+mi}{45}
        \PY{n}{y\PYZus{}4} \PY{o}{=} \PY{o}{.}\PY{l+m+mi}{6}
\end{Verbatim}

    \begin{Verbatim}[commandchars=\\\{\}]
{\color{incolor}In [{\color{incolor}17}]:} \PY{c}{\PYZsh{} Solución}
         
         \PY{c}{\PYZsh{} Coordenadas nodales}
         \PY{n}{coordenadas\PYZus{}nodales} \PY{o}{=} \PY{n}{pd}\PY{o}{.}\PY{n}{DataFrame}\PY{p}{(}\PY{p}{\PYZob{}}\PY{l+s}{\PYZsq{}}\PY{l+s}{X}\PY{l+s}{\PYZsq{}}\PY{p}{:} \PY{p}{[}\PY{n}{x\PYZus{}1}\PY{p}{,} \PY{n}{x\PYZus{}2}\PY{p}{,} \PY{n}{x\PYZus{}3}\PY{p}{]}\PY{p}{,}
                                             \PY{l+s}{\PYZsq{}}\PY{l+s}{Y}\PY{l+s}{\PYZsq{}}\PY{p}{:} \PY{p}{[}\PY{n}{y\PYZus{}1}\PY{p}{,} \PY{n}{y\PYZus{}2}\PY{p}{,} \PY{n}{y\PYZus{}3}\PY{p}{]}\PY{p}{\PYZcb{}}\PY{p}{,}
                                             \PY{n}{index} \PY{o}{=} \PY{p}{[}\PY{l+m+mi}{1}\PY{p}{,}\PY{l+m+mi}{2}\PY{p}{,}\PY{l+m+mi}{3}\PY{p}{]}\PY{p}{)}
         \PY{k}{print}\PY{p}{(}\PY{l+s}{\PYZsq{}}\PY{l+s}{La tabla de coordenadas nodales es la siguiente: }\PY{l+s}{\PYZsq{}}\PY{p}{)}
         \PY{n}{coordenadas\PYZus{}nodales}
\end{Verbatim}

    \begin{Verbatim}[commandchars=\\\{\}]
La tabla de coordenadas nodales es la siguiente:
    \end{Verbatim}

            \begin{Verbatim}[commandchars=\\\{\}]
{\color{outcolor}Out[{\color{outcolor}17}]:}           X        Y
         1  0.00e+00 0.00e+00
         2 -4.50e-01 6.00e-01
         3  8.00e-01 6.00e-01
\end{Verbatim}
        
    \begin{Verbatim}[commandchars=\\\{\}]
{\color{incolor}In [{\color{incolor}18}]:} \PY{c}{\PYZsh{} Conectividad}
         
         \PY{n}{conectividad} \PY{o}{=} \PY{n}{pd}\PY{o}{.}\PY{n}{DataFrame}\PY{p}{(}\PY{p}{\PYZob{}}\PY{l+m+mi}{1}\PY{p}{:} \PY{p}{[}\PY{l+m+mi}{1}\PY{p}{,} \PY{l+m+mi}{1}\PY{p}{,} \PY{l+m+mi}{1}\PY{p}{]}\PY{p}{,}
                                      \PY{l+m+mi}{2}\PY{p}{:} \PY{p}{[}\PY{l+m+mi}{2}\PY{p}{,} \PY{l+m+mi}{3}\PY{p}{,} \PY{l+m+mi}{4}\PY{p}{]}\PY{p}{\PYZcb{}}\PY{p}{,}
                                      \PY{n}{index} \PY{o}{=} \PY{p}{[}\PY{l+m+mi}{1}\PY{p}{,} \PY{l+m+mi}{2}\PY{p}{,} \PY{l+m+mi}{3}\PY{p}{]}\PY{p}{)}
         \PY{k}{print}\PY{p}{(}\PY{l+s}{\PYZsq{}}\PY{l+s}{Tabla de conectividad}\PY{l+s}{\PYZsq{}}\PY{p}{)}
         \PY{n}{conectividad}
\end{Verbatim}

    \begin{Verbatim}[commandchars=\\\{\}]
Tabla de conectividad
    \end{Verbatim}

            \begin{Verbatim}[commandchars=\\\{\}]
{\color{outcolor}Out[{\color{outcolor}18}]:}    1  2
         1  1  2
         2  1  3
         3  1  4
\end{Verbatim}
        
    \begin{Verbatim}[commandchars=\\\{\}]
{\color{incolor}In [{\color{incolor}29}]:} \PY{c}{\PYZsh{} Cosenos directores}
         
         \PY{c}{\PYZsh{} 1}
         \PY{n}{le\PYZus{}1} \PY{o}{=} \PY{n}{np}\PY{o}{.}\PY{n}{sqrt}\PY{p}{(}\PY{p}{(}\PY{n}{x\PYZus{}2} \PY{o}{\PYZhy{}} \PY{n}{x\PYZus{}1}\PY{p}{)} \PY{o}{*}\PY{o}{*} \PY{l+m+mi}{2} \PY{o}{+} \PY{p}{(}\PY{n}{y\PYZus{}2} \PY{o}{\PYZhy{}} \PY{n}{y\PYZus{}1}\PY{p}{)} \PY{o}{*}\PY{o}{*} \PY{l+m+mi}{2}\PY{p}{)}
         \PY{n}{l\PYZus{}1} \PY{o}{=} \PY{p}{(}\PY{n}{x\PYZus{}2} \PY{o}{\PYZhy{}} \PY{n}{x\PYZus{}1}\PY{p}{)} \PY{o}{/} \PY{n}{le\PYZus{}1}
         \PY{n}{m\PYZus{}1} \PY{o}{=} \PY{p}{(}\PY{n}{y\PYZus{}2} \PY{o}{\PYZhy{}} \PY{n}{y\PYZus{}1}\PY{p}{)} \PY{o}{/} \PY{n}{le\PYZus{}1}
         
         \PY{c}{\PYZsh{} 2}
         \PY{n}{le\PYZus{}2} \PY{o}{=} \PY{n}{np}\PY{o}{.}\PY{n}{sqrt}\PY{p}{(}\PY{p}{(}\PY{n}{x\PYZus{}3} \PY{o}{\PYZhy{}} \PY{n}{x\PYZus{}1}\PY{p}{)} \PY{o}{*}\PY{o}{*} \PY{l+m+mi}{2} \PY{o}{+} \PY{p}{(}\PY{n}{y\PYZus{}3} \PY{o}{\PYZhy{}} \PY{n}{y\PYZus{}1}\PY{p}{)} \PY{o}{*}\PY{o}{*} \PY{l+m+mi}{2}\PY{p}{)}
         \PY{n}{l\PYZus{}2} \PY{o}{=} \PY{p}{(}\PY{n}{x\PYZus{}3} \PY{o}{\PYZhy{}} \PY{n}{x\PYZus{}1}\PY{p}{)} \PY{o}{/} \PY{n}{le\PYZus{}2}
         \PY{n}{m\PYZus{}2} \PY{o}{=} \PY{p}{(}\PY{n}{y\PYZus{}3} \PY{o}{\PYZhy{}} \PY{n}{y\PYZus{}1}\PY{p}{)} \PY{o}{/} \PY{n}{le\PYZus{}2}
         
         \PY{c}{\PYZsh{} 3}
         \PY{n}{le\PYZus{}3} \PY{o}{=} \PY{n}{np}\PY{o}{.}\PY{n}{sqrt}\PY{p}{(}\PY{p}{(}\PY{n}{x\PYZus{}4} \PY{o}{\PYZhy{}} \PY{n}{x\PYZus{}1}\PY{p}{)} \PY{o}{*}\PY{o}{*} \PY{l+m+mi}{2} \PY{o}{+} \PY{p}{(}\PY{n}{y\PYZus{}4} \PY{o}{\PYZhy{}} \PY{n}{y\PYZus{}1}\PY{p}{)} \PY{o}{*}\PY{o}{*} \PY{l+m+mi}{2}\PY{p}{)}
         \PY{n}{l\PYZus{}3} \PY{o}{=} \PY{p}{(}\PY{n}{x\PYZus{}4} \PY{o}{\PYZhy{}} \PY{n}{x\PYZus{}1}\PY{p}{)} \PY{o}{/} \PY{n}{le\PYZus{}3}
         \PY{n}{m\PYZus{}3} \PY{o}{=} \PY{p}{(}\PY{n}{y\PYZus{}4} \PY{o}{\PYZhy{}} \PY{n}{y\PYZus{}1}\PY{p}{)} \PY{o}{/} \PY{n}{le\PYZus{}3}
         
         \PY{c}{\PYZsh{} Tabla de cosenos directores}
         \PY{n}{cosenos\PYZus{}dir} \PY{o}{=} \PY{n}{pd}\PY{o}{.}\PY{n}{DataFrame}\PY{p}{(}\PY{p}{\PYZob{}}\PY{l+s}{\PYZsq{}}\PY{l+s}{le}\PY{l+s}{\PYZsq{}}\PY{p}{:} \PY{p}{[}\PY{n}{le\PYZus{}1}\PY{p}{,} \PY{n}{le\PYZus{}2}\PY{p}{,} \PY{n}{le\PYZus{}3}\PY{p}{]}\PY{p}{,}
                                     \PY{l+s}{\PYZsq{}}\PY{l+s}{l}\PY{l+s}{\PYZsq{}} \PY{p}{:} \PY{p}{[}\PY{n}{l\PYZus{}1}\PY{p}{,} \PY{n}{l\PYZus{}2}\PY{p}{,} \PY{n}{l\PYZus{}3}\PY{p}{]}\PY{p}{,}
                                     \PY{l+s}{\PYZsq{}}\PY{l+s}{m}\PY{l+s}{\PYZsq{}} \PY{p}{:} \PY{p}{[}\PY{n}{m\PYZus{}1}\PY{p}{,} \PY{n}{m\PYZus{}2}\PY{p}{,} \PY{n}{m\PYZus{}3}\PY{p}{]}\PY{p}{\PYZcb{}}\PY{p}{,}
                                     \PY{n}{index} \PY{o}{=} \PY{p}{[}\PY{l+m+mi}{1}\PY{p}{,} \PY{l+m+mi}{2}\PY{p}{,} \PY{l+m+mi}{3}\PY{p}{]}\PY{p}{)}
         \PY{k}{print}\PY{p}{(}\PY{l+s}{\PYZsq{}}\PY{l+s}{Tabla de cosenos directores}\PY{l+s}{\PYZsq{}}\PY{p}{)}
         \PY{n}{cosenos\PYZus{}dir}
\end{Verbatim}

    \begin{Verbatim}[commandchars=\\\{\}]
Tabla de cosenos directores
    \end{Verbatim}

            \begin{Verbatim}[commandchars=\\\{\}]
{\color{outcolor}Out[{\color{outcolor}29}]:}           l       le        m
         1 -6.00e-01 7.50e-01 8.00e-01
         2  8.00e-01 1.00e+00 6.00e-01
         3  6.00e-01 7.50e-01 8.00e-01
\end{Verbatim}
        
    \begin{Verbatim}[commandchars=\\\{\}]
{\color{incolor}In [{\color{incolor}5}]:} \PY{c}{\PYZsh{} Matriz de rigidez}
        
        \PY{c}{\PYZsh{} Se define una función que nos calcula la matriz de rigidez}
        \PY{k}{def} \PY{n+nf}{matrizDeRigidez}\PY{p}{(}\PY{n}{E}\PY{p}{,} \PY{n}{A}\PY{p}{,} \PY{n}{le}\PY{p}{,} \PY{n}{l}\PY{p}{,} \PY{n}{m}\PY{p}{,} \PY{n}{i}\PY{p}{,} \PY{n}{j}\PY{p}{)}\PY{p}{:}
            \PY{n}{k} \PY{o}{=} \PY{p}{(}\PY{n}{E} \PY{o}{*} \PY{n}{A}\PY{p}{)} \PY{o}{/} \PY{n}{le}
            \PY{n}{matriz} \PY{o}{=} \PY{n}{k} \PY{o}{*} \PY{n}{np}\PY{o}{.}\PY{n}{array}\PY{p}{(}\PY{p}{[}\PY{p}{[}\PY{n}{l}\PY{o}{*}\PY{o}{*}\PY{l+m+mi}{2}\PY{p}{,} \PY{n}{l}\PY{o}{*}\PY{n}{m}\PY{p}{,} \PY{o}{\PYZhy{}}\PY{n}{l}\PY{o}{*}\PY{o}{*}\PY{l+m+mi}{2}\PY{p}{,} \PY{o}{\PYZhy{}}\PY{n}{l}\PY{o}{*}\PY{n}{m}\PY{p}{]}\PY{p}{,}
                                   \PY{p}{[}\PY{n}{l}\PY{o}{*}\PY{n}{m}\PY{p}{,} \PY{n}{m}\PY{o}{*}\PY{o}{*}\PY{l+m+mi}{2}\PY{p}{,} \PY{o}{\PYZhy{}}\PY{n}{l}\PY{o}{*}\PY{n}{m}\PY{p}{,} \PY{o}{\PYZhy{}}\PY{n}{m}\PY{o}{*}\PY{o}{*}\PY{l+m+mi}{2}\PY{p}{]}\PY{p}{,}
                                   \PY{p}{[}\PY{o}{\PYZhy{}}\PY{n}{l}\PY{o}{*}\PY{o}{*}\PY{l+m+mi}{2}\PY{p}{,} \PY{o}{\PYZhy{}}\PY{n}{l}\PY{o}{*}\PY{n}{m}\PY{p}{,} \PY{n}{l}\PY{o}{*}\PY{o}{*}\PY{l+m+mi}{2}\PY{p}{,} \PY{n}{l}\PY{o}{*}\PY{n}{m}\PY{p}{]}\PY{p}{,}
                                   \PY{p}{[}\PY{o}{\PYZhy{}}\PY{n}{l}\PY{o}{*}\PY{n}{m}\PY{p}{,} \PY{o}{\PYZhy{}}\PY{n}{m}\PY{o}{*}\PY{o}{*}\PY{l+m+mi}{2}\PY{p}{,} \PY{n}{l}\PY{o}{*}\PY{n}{m}\PY{p}{,} \PY{n}{m}\PY{o}{*}\PY{o}{*}\PY{l+m+mi}{2}\PY{p}{]}\PY{p}{]}\PY{p}{)}
            \PY{n}{tabla} \PY{o}{=} \PY{n}{pd}\PY{o}{.}\PY{n}{DataFrame}\PY{p}{(}\PY{n}{matriz}\PY{p}{,} \PY{n}{columns} \PY{o}{=} \PY{p}{[}\PY{l+m+mi}{2}\PY{o}{*}\PY{n}{i}\PY{o}{\PYZhy{}}\PY{l+m+mi}{1}\PY{p}{,} \PY{l+m+mi}{2}\PY{o}{*}\PY{n}{i}\PY{p}{,} \PY{l+m+mi}{2}\PY{o}{*}\PY{n}{j}\PY{o}{\PYZhy{}}\PY{l+m+mi}{1}\PY{p}{,} \PY{l+m+mi}{2}\PY{o}{*}\PY{n}{j}\PY{p}{]}\PY{p}{,}\PY{n}{index} \PY{o}{=} \PY{p}{[}\PY{l+m+mi}{2}\PY{o}{*}\PY{n}{i}\PY{o}{\PYZhy{}}\PY{l+m+mi}{1}\PY{p}{,} \PY{l+m+mi}{2}\PY{o}{*}\PY{n}{i}\PY{p}{,} \PY{l+m+mi}{2}\PY{o}{*}\PY{n}{j}\PY{o}{\PYZhy{}}\PY{l+m+mi}{1}\PY{p}{,} \PY{l+m+mi}{2}\PY{o}{*}\PY{n}{j}\PY{p}{]}\PY{p}{)}
            \PY{k}{return} \PY{n}{tabla}
        
        \PY{c}{\PYZsh{} 1}
        \PY{n}{matriz\PYZus{}de\PYZus{}rig\PYZus{}1} \PY{o}{=} \PY{n}{matrizDeRigidez}\PY{p}{(}\PY{n}{E}\PY{p}{,} \PY{n}{A}\PY{p}{,} \PY{n}{le\PYZus{}1}\PY{p}{,} \PY{n}{l\PYZus{}1}\PY{p}{,} \PY{n}{m\PYZus{}1}\PY{p}{,} \PY{l+m+mi}{1}\PY{p}{,} \PY{l+m+mi}{2}\PY{p}{)}
        \PY{k}{print}\PY{p}{(}\PY{l+s}{\PYZsq{}}\PY{l+s}{La matriz de rigidez 1 es: }\PY{l+s}{\PYZsq{}}\PY{p}{)}
        \PY{n}{matriz\PYZus{}de\PYZus{}rig\PYZus{}1}
\end{Verbatim}

    \begin{Verbatim}[commandchars=\\\{\}]
La matriz de rigidez 1 es:
    \end{Verbatim}

            \begin{Verbatim}[commandchars=\\\{\}]
{\color{outcolor}Out[{\color{outcolor}5}]:}           1                2         3                4
        1  24000000 -32000000.000000 -24000000  32000000.000000
        2 -32000000  42666666.666667  32000000 -42666666.666667
        3 -24000000  32000000.000000  24000000 -32000000.000000
        4  32000000 -42666666.666667 -32000000  42666666.666667
\end{Verbatim}
        
    \begin{Verbatim}[commandchars=\\\{\}]
{\color{incolor}In [{\color{incolor}6}]:} \PY{c}{\PYZsh{} 2}
        \PY{n}{matriz\PYZus{}de\PYZus{}rig\PYZus{}2} \PY{o}{=} \PY{n}{matrizDeRigidez}\PY{p}{(}\PY{n}{E}\PY{p}{,} \PY{n}{A}\PY{p}{,} \PY{n}{le\PYZus{}2}\PY{p}{,} \PY{n}{l\PYZus{}2}\PY{p}{,} \PY{n}{m\PYZus{}2}\PY{p}{,} \PY{l+m+mi}{1}\PY{p}{,} \PY{l+m+mi}{3}\PY{p}{)}
        \PY{k}{print}\PY{p}{(}\PY{l+s}{\PYZsq{}}\PY{l+s}{La matriz de rigidez 2 es: }\PY{l+s}{\PYZsq{}}\PY{p}{)}
        \PY{n}{matriz\PYZus{}de\PYZus{}rig\PYZus{}2}
\end{Verbatim}

    \begin{Verbatim}[commandchars=\\\{\}]
La matriz de rigidez 2 es:
    \end{Verbatim}

            \begin{Verbatim}[commandchars=\\\{\}]
{\color{outcolor}Out[{\color{outcolor}6}]:}           1         2         5         6
        1  32000000  24000000 -32000000 -24000000
        2  24000000  18000000 -24000000 -18000000
        5 -32000000 -24000000  32000000  24000000
        6 -24000000 -18000000  24000000  18000000
\end{Verbatim}
        
    \begin{Verbatim}[commandchars=\\\{\}]
{\color{incolor}In [{\color{incolor}7}]:} \PY{c}{\PYZsh{} 3}
        \PY{n}{matriz\PYZus{}de\PYZus{}rig\PYZus{}3} \PY{o}{=} \PY{n}{matrizDeRigidez}\PY{p}{(}\PY{n}{E}\PY{p}{,} \PY{n}{A}\PY{p}{,} \PY{n}{le\PYZus{}3}\PY{p}{,} \PY{n}{l\PYZus{}3}\PY{p}{,} \PY{n}{m\PYZus{}3}\PY{p}{,} \PY{l+m+mi}{1}\PY{p}{,} \PY{l+m+mi}{4}\PY{p}{)}
        \PY{k}{print}\PY{p}{(}\PY{l+s}{\PYZsq{}}\PY{l+s}{La matriz de rigidez 3 es: }\PY{l+s}{\PYZsq{}}\PY{p}{)}
        \PY{n}{matriz\PYZus{}de\PYZus{}rig\PYZus{}3}
\end{Verbatim}

    \begin{Verbatim}[commandchars=\\\{\}]
La matriz de rigidez 3 es:
    \end{Verbatim}

            \begin{Verbatim}[commandchars=\\\{\}]
{\color{outcolor}Out[{\color{outcolor}7}]:}           1                2         7                8
        1  24000000  32000000.000000 -24000000 -32000000.000000
        2  32000000  42666666.666667 -32000000 -42666666.666667
        7 -24000000 -32000000.000000  24000000  32000000.000000
        8 -32000000 -42666666.666667  32000000  42666666.666667
\end{Verbatim}
        
    \begin{Verbatim}[commandchars=\\\{\}]
{\color{incolor}In [{\color{incolor}8}]:} \PY{c}{\PYZsh{} Se ubican las matrices de rigidez de los elementos en la matriz de rigidez estructural}
        
        \PY{c}{\PYZsh{} Creamos una matriz 8x8 llena de zeros}
        \PY{n}{matriz\PYZus{}de\PYZus{}rigidez\PYZus{}estr} \PY{o}{=} \PY{n}{pd}\PY{o}{.}\PY{n}{DataFrame}\PY{p}{(}\PY{n}{np}\PY{o}{.}\PY{n}{zeros}\PY{p}{(}\PY{p}{(}\PY{l+m+mi}{8}\PY{p}{,}\PY{l+m+mi}{8}\PY{p}{)}\PY{p}{)}\PY{p}{,} \PY{n}{index} \PY{o}{=} \PY{p}{[}\PY{l+m+mi}{1}\PY{p}{,}\PY{l+m+mi}{2}\PY{p}{,}\PY{l+m+mi}{3}\PY{p}{,}\PY{l+m+mi}{4}\PY{p}{,}\PY{l+m+mi}{5}\PY{p}{,}\PY{l+m+mi}{6}\PY{p}{,}\PY{l+m+mi}{7}\PY{p}{,}\PY{l+m+mi}{8}\PY{p}{]}\PY{p}{,} \PY{n}{columns} \PY{o}{=} \PY{p}{[}\PY{l+m+mi}{1}\PY{p}{,}\PY{l+m+mi}{2}\PY{p}{,}\PY{l+m+mi}{3}\PY{p}{,}\PY{l+m+mi}{4}\PY{p}{,}\PY{l+m+mi}{5}\PY{p}{,}\PY{l+m+mi}{6}\PY{p}{,}\PY{l+m+mi}{7}\PY{p}{,}\PY{l+m+mi}{8}\PY{p}{]}\PY{p}{)}
        
        \PY{c}{\PYZsh{} Ubicamos la matriz del elemento 1}
        \PY{k}{for} \PY{n}{i} \PY{o+ow}{in} \PY{n}{matriz\PYZus{}de\PYZus{}rig\PYZus{}1}\PY{o}{.}\PY{n}{index}\PY{p}{:}
            \PY{k}{for} \PY{n}{j} \PY{o+ow}{in} \PY{n}{matriz\PYZus{}de\PYZus{}rig\PYZus{}1}\PY{o}{.}\PY{n}{columns}\PY{p}{:}
                \PY{n}{matriz\PYZus{}de\PYZus{}rigidez\PYZus{}estr}\PY{o}{.}\PY{n}{loc}\PY{p}{[}\PY{n}{i}\PY{p}{,}\PY{n}{j}\PY{p}{]} \PY{o}{=} \PY{n}{matriz\PYZus{}de\PYZus{}rigidez\PYZus{}estr}\PY{o}{.}\PY{n}{loc}\PY{p}{[}\PY{n}{i}\PY{p}{,}\PY{n}{j}\PY{p}{]} \PY{o}{+} \PY{n}{matriz\PYZus{}de\PYZus{}rig\PYZus{}1}\PY{o}{.}\PY{n}{loc}\PY{p}{[}\PY{n}{i}\PY{p}{,}\PY{n}{j}\PY{p}{]}
        
        \PY{c}{\PYZsh{} 2}
        \PY{k}{for} \PY{n}{i} \PY{o+ow}{in} \PY{n}{matriz\PYZus{}de\PYZus{}rig\PYZus{}2}\PY{o}{.}\PY{n}{index}\PY{p}{:}
            \PY{k}{for} \PY{n}{j} \PY{o+ow}{in} \PY{n}{matriz\PYZus{}de\PYZus{}rig\PYZus{}2}\PY{o}{.}\PY{n}{columns}\PY{p}{:}
                \PY{n}{matriz\PYZus{}de\PYZus{}rigidez\PYZus{}estr}\PY{o}{.}\PY{n}{loc}\PY{p}{[}\PY{n}{i}\PY{p}{,}\PY{n}{j}\PY{p}{]} \PY{o}{=} \PY{n}{matriz\PYZus{}de\PYZus{}rigidez\PYZus{}estr}\PY{o}{.}\PY{n}{loc}\PY{p}{[}\PY{n}{i}\PY{p}{,}\PY{n}{j}\PY{p}{]} \PY{o}{+} \PY{n}{matriz\PYZus{}de\PYZus{}rig\PYZus{}2}\PY{o}{.}\PY{n}{loc}\PY{p}{[}\PY{n}{i}\PY{p}{,}\PY{n}{j}\PY{p}{]}
        
        \PY{c}{\PYZsh{} 3}
        \PY{k}{for} \PY{n}{i} \PY{o+ow}{in} \PY{n}{matriz\PYZus{}de\PYZus{}rig\PYZus{}3}\PY{o}{.}\PY{n}{index}\PY{p}{:}
            \PY{k}{for} \PY{n}{j} \PY{o+ow}{in} \PY{n}{matriz\PYZus{}de\PYZus{}rig\PYZus{}3}\PY{o}{.}\PY{n}{columns}\PY{p}{:}
                \PY{n}{matriz\PYZus{}de\PYZus{}rigidez\PYZus{}estr}\PY{o}{.}\PY{n}{loc}\PY{p}{[}\PY{n}{i}\PY{p}{,}\PY{n}{j}\PY{p}{]} \PY{o}{=} \PY{n}{matriz\PYZus{}de\PYZus{}rigidez\PYZus{}estr}\PY{o}{.}\PY{n}{loc}\PY{p}{[}\PY{n}{i}\PY{p}{,}\PY{n}{j}\PY{p}{]} \PY{o}{+} \PY{n}{matriz\PYZus{}de\PYZus{}rig\PYZus{}3}\PY{o}{.}\PY{n}{loc}\PY{p}{[}\PY{n}{i}\PY{p}{,}\PY{n}{j}\PY{p}{]}
\end{Verbatim}

    \begin{Verbatim}[commandchars=\\\{\}]
{\color{incolor}In [{\color{incolor}9}]:} \PY{c}{\PYZsh{} Encendemos la notación científica con 2 decimales}
        \PY{n}{pd}\PY{o}{.}\PY{n}{options}\PY{o}{.}\PY{n}{display}\PY{o}{.}\PY{n}{float\PYZus{}format} \PY{o}{=} \PY{l+s}{\PYZsq{}}\PY{l+s}{\PYZob{}:,.2e\PYZcb{}}\PY{l+s}{\PYZsq{}}\PY{o}{.}\PY{n}{format}
        
        \PY{c}{\PYZsh{} Mostramos la matriz}
        \PY{n}{matriz\PYZus{}de\PYZus{}rigidez\PYZus{}estr}
\end{Verbatim}

            \begin{Verbatim}[commandchars=\\\{\}]
{\color{outcolor}Out[{\color{outcolor}9}]:}           1         2         3         4         5         6         7  \textbackslash{}
        1  8.00e+07  2.40e+07 -2.40e+07  3.20e+07 -3.20e+07 -2.40e+07 -2.40e+07   
        2  2.40e+07  1.03e+08  3.20e+07 -4.27e+07 -2.40e+07 -1.80e+07 -3.20e+07   
        3 -2.40e+07  3.20e+07  2.40e+07 -3.20e+07  0.00e+00  0.00e+00  0.00e+00   
        4  3.20e+07 -4.27e+07 -3.20e+07  4.27e+07  0.00e+00  0.00e+00  0.00e+00   
        5 -3.20e+07 -2.40e+07  0.00e+00  0.00e+00  3.20e+07  2.40e+07  0.00e+00   
        6 -2.40e+07 -1.80e+07  0.00e+00  0.00e+00  2.40e+07  1.80e+07  0.00e+00   
        7 -2.40e+07 -3.20e+07  0.00e+00  0.00e+00  0.00e+00  0.00e+00  2.40e+07   
        8 -3.20e+07 -4.27e+07  0.00e+00  0.00e+00  0.00e+00  0.00e+00  3.20e+07   
        
                  8  
        1 -3.20e+07  
        2 -4.27e+07  
        3  0.00e+00  
        4  0.00e+00  
        5  0.00e+00  
        6  0.00e+00  
        7  3.20e+07  
        8  4.27e+07  
\end{Verbatim}
        
    \begin{Verbatim}[commandchars=\\\{\}]
{\color{incolor}In [{\color{incolor}10}]:} \PY{c}{\PYZsh{} los grados de libertad corresponden a 1 y 2asi que eliminamos los demás}
         \PY{n}{matriz\PYZus{}k\PYZus{}tabla} \PY{o}{=} \PY{n}{matriz\PYZus{}de\PYZus{}rigidez\PYZus{}estr}\PY{o}{.}\PY{n}{loc}\PY{p}{[}\PY{l+m+mi}{0}\PY{p}{:}\PY{l+m+mi}{2}\PY{p}{,}\PY{p}{[}\PY{l+m+mi}{1}\PY{p}{,}\PY{l+m+mi}{2}\PY{p}{]}\PY{p}{]}
         \PY{n}{matriz\PYZus{}k\PYZus{}tabla}
\end{Verbatim}

            \begin{Verbatim}[commandchars=\\\{\}]
{\color{outcolor}Out[{\color{outcolor}10}]:}          1        2
         1 8.00e+07 2.40e+07
         2 2.40e+07 1.03e+08
\end{Verbatim}
        
    \begin{Verbatim}[commandchars=\\\{\}]
{\color{incolor}In [{\color{incolor}21}]:} \PY{c}{\PYZsh{} se ha trabajo con tablas, ahora se lleva realmente a un array o matriz}
         
         \PY{c}{\PYZsh{} creamos una matriz 2x2 de ceros}
         \PY{n}{matriz\PYZus{}k} \PY{o}{=} \PY{n}{np}\PY{o}{.}\PY{n}{zeros}\PY{p}{(}\PY{p}{[}\PY{l+m+mi}{2}\PY{p}{,}\PY{l+m+mi}{2}\PY{p}{]}\PY{p}{)}
         
         \PY{c}{\PYZsh{} se llen con las filas y columnas}
         \PY{k}{for} \PY{n}{i} \PY{o+ow}{in} \PY{n}{matriz\PYZus{}k\PYZus{}tabla}\PY{o}{.}\PY{n}{index}\PY{p}{:}
             \PY{k}{for} \PY{n}{j} \PY{o+ow}{in} \PY{n}{matriz\PYZus{}k\PYZus{}tabla}\PY{o}{.}\PY{n}{columns}\PY{p}{:}
                 \PY{n}{matriz\PYZus{}k}\PY{p}{[}\PY{n}{i}\PY{o}{\PYZhy{}}\PY{l+m+mi}{1}\PY{p}{,}\PY{n}{j}\PY{o}{\PYZhy{}}\PY{l+m+mi}{1}\PY{p}{]} \PY{o}{=} \PY{n}{matriz\PYZus{}k\PYZus{}tabla}\PY{o}{.}\PY{n}{loc}\PY{p}{[}\PY{n}{i}\PY{p}{,}\PY{n}{j}\PY{p}{]}
                 
         \PY{c}{\PYZsh{} mostramos la matriz}
         \PY{k}{print}\PY{p}{(}\PY{n}{matriz\PYZus{}k}\PY{p}{)}
\end{Verbatim}

    \begin{Verbatim}[commandchars=\\\{\}]
[[  8.00000000e+07   2.40000000e+07]
 [  2.40000000e+07   1.03333333e+08]]
    \end{Verbatim}

    \begin{Verbatim}[commandchars=\\\{\}]
{\color{incolor}In [{\color{incolor}25}]:} \PY{c}{\PYZsh{} para resolver importamos el módulo solve de la librería scipy.linag}
         \PY{c}{\PYZsh{} estos módulos son programados a bajo nivel con lenguajes como C o Fortran}
         \PY{k+kn}{from} \PY{n+nn}{scipy.linalg} \PY{k+kn}{import} \PY{n}{solve}
         
         \PY{c}{\PYZsh{} creamos la matriz de fuerza}
         \PY{n}{f} \PY{o}{=} \PY{n}{np}\PY{o}{.}\PY{n}{array}\PY{p}{(}\PY{p}{[}\PY{l+m+mi}{0}\PY{p}{,} \PY{o}{\PYZhy{}}\PY{n}{F}\PY{p}{]}\PY{p}{)}
         
         \PY{c}{\PYZsh{} resolvemos el problema F = kU obteniendo el desplazamiento}
         \PY{n}{desp} \PY{o}{=} \PY{n}{solve}\PY{p}{(}\PY{n}{matriz\PYZus{}k}\PY{p}{,}\PY{n}{f}\PY{p}{)}
         \PY{k}{print}\PY{p}{(}\PY{l+s}{\PYZsq{}}\PY{l+s}{El desplazamiento en el punto 1 es [m]:}\PY{l+s}{\PYZsq{}}\PY{p}{)}
         \PY{k}{print}\PY{p}{(}\PY{l+s}{\PYZsq{}}\PY{l+s}{En el eje x}\PY{l+s}{\PYZsq{}}\PY{p}{,} \PY{n}{desp}\PY{p}{[}\PY{l+m+mi}{0}\PY{p}{]}\PY{p}{)}
         \PY{k}{print}\PY{p}{(}\PY{l+s}{\PYZsq{}}\PY{l+s}{En el eje y}\PY{l+s}{\PYZsq{}}\PY{p}{,} \PY{n}{desp}\PY{p}{[}\PY{l+m+mi}{1}\PY{p}{]}\PY{p}{)}
\end{Verbatim}

    \begin{Verbatim}[commandchars=\\\{\}]
El desplazamiento en el punto 1 es [m]:
En el eje x 5.61719833564e-05
En el eje y -0.000187239944521
    \end{Verbatim}

    \begin{Verbatim}[commandchars=\\\{\}]
{\color{incolor}In [{\color{incolor}28}]:} \PY{c}{\PYZsh{} esfuerzo en 3}
         \PY{n}{a} \PY{o}{=} \PY{n}{np}\PY{o}{.}\PY{n}{array}\PY{p}{(}\PY{p}{[}\PY{o}{\PYZhy{}}\PY{n}{l\PYZus{}3}\PY{p}{,} \PY{o}{\PYZhy{}}\PY{n}{m\PYZus{}3}\PY{p}{,} \PY{n}{l\PYZus{}3}\PY{p}{,} \PY{n}{m\PYZus{}3}\PY{p}{]}\PY{p}{)}
         \PY{n}{q} \PY{o}{=} \PY{n}{np}\PY{o}{.}\PY{n}{array}\PY{p}{(}\PY{p}{[}\PY{n}{desp}\PY{p}{[}\PY{l+m+mi}{0}\PY{p}{]}\PY{p}{,} \PY{n}{desp}\PY{p}{[}\PY{l+m+mi}{1}\PY{p}{]}\PY{p}{,} \PY{l+m+mi}{0}\PY{p}{,} \PY{l+m+mi}{0}\PY{p}{]}\PY{p}{)}
         \PY{n}{sigma\PYZus{}3} \PY{o}{=} \PY{p}{(}\PY{n}{E} \PY{o}{/} \PY{n}{le\PYZus{}3}\PY{p}{)} \PY{o}{*} \PY{n}{np}\PY{o}{.}\PY{n}{dot}\PY{p}{(}\PY{n}{a}\PY{p}{,}\PY{n}{q}\PY{p}{)}
         \PY{k}{print}\PY{p}{(}\PY{l+s}{\PYZsq{}}\PY{l+s}{El esfuerzo en 3 es [Pa]:}\PY{l+s}{\PYZsq{}}\PY{p}{,}\PY{n}{sigma\PYZus{}3}\PY{p}{)}
\end{Verbatim}

    \begin{Verbatim}[commandchars=\\\{\}]
El esfuerzo en 3 es [Pa]: 30957004.1609
    \end{Verbatim}

    \begin{Verbatim}[commandchars=\\\{\}]
{\color{incolor}In [{\color{incolor} }]:} 
\end{Verbatim}


    % Add a bibliography block to the postdoc
    
    
    
    \end{document}
